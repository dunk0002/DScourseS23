% Fonts/languages
\documentclass[12pt,english]{exam}
\IfFileExists{lmodern.sty}{\usepackage{lmodern}}{}
\usepackage[T1]{fontenc}
\usepackage[latin9]{inputenc}
\usepackage{babel}
\usepackage{mathpazo}
%\usepackage{mathptmx}

% Colors: see  http://www.math.umbc.edu/~rouben/beamer/quickstart-Z-H-25.html
\usepackage{color}
\usepackage[dvipsnames]{xcolor}
\definecolor{byublue}     {RGB}{0.  ,30. ,76. }
\definecolor{deepred}     {RGB}{190.,0.  ,0.  }
\definecolor{deeperred}   {RGB}{160.,0.  ,0.  }
\newcommand{\textblue}[1]{\textcolor{byublue}{#1}}
\newcommand{\textred}[1]{\textcolor{deeperred}{#1}}

% Layout
\usepackage{setspace} %singlespacing; onehalfspacing; doublespacing; setstretch{1.1}
\setstretch{1.2}
\usepackage[verbose,nomarginpar,margin=1in]{geometry} % Margins
\setlength{\headheight}{15pt} % Sufficent room for headers
\usepackage[bottom]{footmisc} % Forces footnotes on bottom

% Headers/Footers
\setlength{\headheight}{15pt}   
%\usepackage{fancyhdr}
%\pagestyle{fancy}
%\lhead{For-Profit Notes} \chead{} \rhead{\thepage}
%\lfoot{} \cfoot{} \rfoot{}

% Useful Packages
%\usepackage{bookmark} % For speedier bookmarks
\usepackage{amsthm}   % For detailed theorems
\usepackage{amssymb}  % For fancy math symbols
\usepackage{amsmath}  % For awesome equations/equation arrays
\usepackage{array}    % For tubular tables
\usepackage{longtable}% For long tables
\usepackage[flushleft]{threeparttable} % For three-part tables
\usepackage{multicol} % For multi-column cells
\usepackage{graphicx} % For shiny pictures
\usepackage{subfig}   % For sub-shiny pictures
\usepackage{enumerate}% For cusomtizable lists
\usepackage{pstricks,pst-node,pst-tree,pst-plot} % For trees

% Bib
\usepackage[authoryear]{natbib} % Bibliography
\usepackage{url}                % Allows urls in bib

% TOC
\setcounter{tocdepth}{4}

% Links
\usepackage{hyperref}    % Always add hyperref (almost) last
\hypersetup{colorlinks,breaklinks,citecolor=black,filecolor=black,linkcolor=byublue,urlcolor=blue,pdfstartview={FitH}}
\usepackage[all]{hypcap} % Links point to top of image, builds on hyperref
\usepackage{breakurl}    % Allows urls to wrap, including hyperref

\pagestyle{head}
\firstpageheader{\textbf{\class\ - \term}}{\textbf{\examnum}}{\textbf{Due: Feb. 7\\ beginning of class}}
\runningheader{\textbf{\class\ - \term}}{\textbf{\examnum}}{\textbf{Due: Feb. 7\\ beginning of class}}
\runningheadrule

\newcommand{\class}{Econ 5253}
\newcommand{\term}{Spring 2023}
\newcommand{\examdate}{Due: February 7, 2023}
% \newcommand{\timelimit}{30 Minutes}

\noprintanswers                         % Uncomment for no solutions version
\newcommand{\examnum}{Problem Set 2}           % Uncomment for no solutions version
% \printanswers                           % Uncomment for solutions version
% \newcommand{\examnum}{Problem Set 2 - Solutions} % Uncomment for solutions version

\begin{document}
This problem set will provide an opportunity for you to practice accessing computing resources on the OSCER server here at OU.

As with the previous problem set, you will submit this problem set by pushing the document to \emph{your} (private) fork of the class repository. You will put this and all other problem sets in the path \texttt{/DScourseS23/ProblemSets/PS2/} and name the file \texttt{PS2\_LastName.pdf}.
\begin{questions}
\question Make sure you are able to access OSCER.

\question Setting up SSH with GitHub on OSCER.

Please follow these steps closely to allow SSH authentication on OSCER:

\begin{enumerate}
    \item \texttt{ssh} into OSCER from your RStudio terminal
    \item at the bash prompt, type \texttt{cd\ .ssh}
    \begin{itemize}
        \item If you get an error that that directory doesn't exist, then type \texttt{mkdir\ .ssh}, followed by \texttt{cd\ .ssh} 
    \end{itemize}
    \item type \texttt{ssh-keygen -t rsa -b 4096 -C "your.email@address.com"} where you fill in with the email address you use for GitHub
    \begin{itemize}
        \item when it asks you for the file in which to save the key, label it ``ghDS23''
        \item you may enter a passphrase to protect the ssh key but it is not required; you can simply hit enter twice
    \end{itemize}
    \item it should tell you ``Your identification has been saved in ghDS23'' and ``Your public key has been saved in ghDS23.pub'' and then print out some other stuff about a fingerprint and randomart image.
    \item type \texttt{cd \~}
    \item type \texttt{eval "\$(ssh-agent -s)"}
    \item type \texttt{ssh-add \~/.ssh/ghDS23}
    \item in a web browser, visit \url{https://github.com/settings/keys} and click "new SSH key". Name it "OSCER"
    \item Copy the contents of your \texttt{ghDS23.pub} file and paste it into the key box. You can do this in a couple of ways:
    \begin{itemize}
        \item type \texttt{cat \~/.ssh/ghDS23.pub} and then select all of the text with your mouse cursor. In some terminals, this also copies the text. Otherwise you can try and right-click and then copy it. Then paste it into the box on the GitHub SSH key webpage.
        \item email the key to yourself by typing
        
        \texttt{mail -s "public key" email@address.com < ~/.ssh/ghDS23.pub}

        Then copy/paste it into the box on the GitHub SSH key webpage
    \end{itemize}
\end{enumerate}

    If all of that was too much for you, you can also check out this YouTube video for a walkthrough: \url{https://www.youtube.com/watch?v=s6KTbytdNgs}

\question If you have already cloned your forked GitHub repository to your home directory on OSCER, you will need to adjust the URL for the remote by following these steps:
\begin{enumerate}
    \item Make sure you are in your local repository directory by typing \texttt{cd \~/DScourseS23}.
    \item Type 
        
        \texttt{git remote set-url origin git@github.com:[your-github-username]/DScourseS23.git}

        Note that you can also git this url from the same place that you get the clone on your github fork. Just go to the "SSH" instead of "HHTPS" section.
    \item Do a \texttt{git pull} and then a \texttt{git push} to make sure things work well with the new SSH tokens.
\end{enumerate}

\question If you have not already, clone your forked GitHub repository to your home directory on OSCER by following these steps:
\begin{enumerate}
    \item Make sure you are in your home directory by typing \texttt{cd \~}.
    \item Type \texttt{git clone git@github.com:[your-github-username]/DScourseS23.git}
    \item Check that everything works by typing \texttt{ls} and hitting enter. You should see a directory called ``DScourseS23'' and if you change to that directory, you should see an identical directory and file structure to what is on your GitHub private fork of the course repo.
\end{enumerate}

\question Go to \url{www.overleaf.com} and create another .tex document, this time naming it \texttt{PS2\_LastName.tex}. In it, write down a list (in a LaTeX \texttt{itemize} envrionment) that outlines the main tools of a data scientist (as discussed in class). Compile the PDF.

\question Compile your .tex file, download the PDF and .tex file, and transfer it to your cloned repository on OSCER using your SFTP client of choice, or via the \texttt{scp} command at the Terminal (in RStudio or elsewhere). If you're using FileZilla, you can set up your connection by clicking File $\rightarrow$ Site Manager $\rightarrow$ and then clicking on ``New Site'' and entering in ``OSCER'' as the name (instead of ``New Site''), ``dtn2.oscer.ou.edu'' as the Host, ``SFTP'' as the protocol, and your username as the user. It's not a good idea to store passwords on your machine, so leave that field blank and make sure that under ``logon type'' you click ``ask for password.'' If you are using something else besides FileZilla, there should be a similar ``site manager'' type of interface for you to enter in your credentials. Consult the help resources for your specific software.

Note: you may also do the following. From the terminal in RStudio, navigate to a temporary directory on your computer (e.g. Downloads folder). Then type (all on one line) \texttt{scp -p PS2\_LastName.tex}\par \texttt{username@schooner.oscer.ou.edu:/home/username/DScourseS23/ProblemSets/PS2/} (and the same for your \texttt{.pdf} file). Do \textbf{not} put these files in your fork on your personal laptop; otherwise git will detect a merge conflict and that will be a painful process to resolve.

\question Make sure that your .tex and .pdf files have the correct naming convention (see top of this problem set for directions) and are located in the correct directory. If the directory does not exist, create it using the \texttt{mkdir} command.

\question Update your OSCER git repository (in your OSCER home directory) by using the following commands:
\begin{itemize}
    \item \texttt{cd \~}
    \item \texttt{cd DScourseS23/ProblemSets/PS2} (note: you may need to create the \texttt{PS2} directory if it doesn't exist)
    \item \texttt{git add PS2\_LastName.*}
    \item \texttt{git commit -m 'Completed PS2'}
    \item \texttt{git push origin master}
\end{itemize}
Once you have done this, issue a \texttt{git pull} from the location of your other local git repository (e.g. on RStudio on your personal computer). Verify that the PS2 directory appears in the appropriate place in your other local repository.

\question Make sure that you have R version 4.0.2 (or higher) installed on OSCER. To do this, take the following steps:
\begin{enumerate}
    \item At the command prompt, type \texttt{nano \textasciitilde/.bash\_profile} and press enter. 
    \item Type \texttt{module load R/4.0.2-foss-2020a} somewhere in the middle of that file. 
    \item At the bottom of the file, type the following:
        \begin{itemize}
        \item type \texttt{eval "\$(ssh-agent -s)"}
        \item type \texttt{ssh-add \~/.ssh/ghDS23}
        \end{itemize}
    \item Then push Control+X and then type ``yes'' (or ``y'') when it asks if you want to modify the file.
\end{enumerate}

\question Now open R (on OSCER), verify that it is running the version mentioned above, and install the \texttt{xml2} package by typing \texttt{install.packages('xml2')} at the command prompt. (\textbf{NOTE: The package is ``X-M-L 2,'' NOT ``X-M 12''}) Answer `yes' to install it in your home directory. Enter the number of any of the server mirrors that come up (I usually choose Texas). Building the package may take a couple of minutes. Once that has finished, repeat the process, but substitute \texttt{tidyverse} for \texttt{xml2}. Building \texttt{tidyverse} may take as much as 10-15 minutes. You don't need to watch it build, but be sure not to close your SSH terminal window or otherwise disconnect your session. When you are done with the build, exit R and log off of OSCER. You can always exit a program (or logout) on OSCER by typing Control+D.

\question Update your fork of the class repository on both your personal laptop as well as on OSCER, since I'm been continuously updating things (and as part of PS1, everyone in the class submitted pull requests). The command to do this is \texttt{git pull upstream master}. A step-by-step help for how to do this is located \href{https://help.github.com/articles/syncing-a-fork/}{here}. More simply, you may also just go to your fork on GitHub and click the button that says ``Sync fork'' or ``Fetch upstream'' or similar.

\end{questions}
\end{document}
